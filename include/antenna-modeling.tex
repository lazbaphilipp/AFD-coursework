\chap{Приложение A: Моделирование излучателя}
\setcounter{chapter}{1}
\setcounter{figure}{0}
\renewcommand\thechapter{\Alph{chapter}}

Рассчитаем антенну с помощью Keysight EmPRO. Диаграмму направленности в плоскостях X,Y можно увидеть на Рис. \ref{fig:simulation-results}.

\begin{figure}[H]
	\centering
	\begin{tikzpicture}[scale=1.8]
		\begin{polaraxis}[
			xticklabel=$\pgfmathprintnumber{\tick}^\circ$,
			xtick={0,30,60,...,330},
			ytick={-15,-10,-5,0,7.5,10.5,15},
			ymin=-15, ymax=15,
			y coord trafo/.code=\pgfmathparse{#1+15},
			rotate=-90,
			y coord inv trafo/.code=\pgfmathparse{#1-15},
			x dir=reverse,
			xticklabel style={anchor=-\tick-90},
			yticklabel style={anchor=east, xshift=-4.75cm},
			y axis line style={yshift=-4.75cm},
			ytick style={yshift=-4.75cm}
			]
			\addplot [no markers, thick, blue] table [col sep=comma,x=ang, y=gain0] {simulation-results/res.dat};
			\addlegendentry{$\varphi=0^\circ$};
			\addplot [no markers, thick, red] table [col sep=comma,x=ang, y=gain90] {simulation-results/res.dat};
			\addlegendentry{$\varphi=90^\circ$};
			
			\node[pin={[pin edge={thin},pin distance=3ex]190:{\parbox{2em}{\raggedleft $-15^\circ,$ \\ $-12^\circ$} }}] at (345,7.5) {};
			\node[pin={[pin edge={thin},pin distance=0.1ex]290:{$\theta=25^\circ$}}] at (25,7.5) {};
			\node[pin={[pin edge={thin},pin distance=0.1ex]260:{$\theta=20^\circ$}}] at (20,7.5) {};
			\node at (5,12.5) {$\theta=5^\circ$};
			\addplot [only marks] table {
				20 7.5
				25 7.5
				345 7.5
				5 10.5
			};
		\end{polaraxis}
	\end{tikzpicture}
	\caption{ Диаграмма направленности рассчётной антенны}
	\label{fig:simulation-results}
\end{figure}

По графику видно, что отклонение главного луча составило $5^\circ$ и ширина луча по оХ и оУ равна соответственно $40^\circ$ и $32^\circ$, что соотносится с требованием $\theta_\text{ск}= \pm18^\circ$ из ТЗ, однако заднее излучение слишком сильное, что говорит о том, что при моделировании был выбран нелостаточно большой экран.