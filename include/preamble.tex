%% Working with russian language
\usepackage{cmap}
\usepackage{mathtext}
\usepackage[T2A]{fontenc}
\usepackage[utf8]{inputenc}
\usepackage[english,russian]{babel}
\usepackage{indentfirst}
\frenchspacing
\usepackage{cancel}

	
\usepackage[normalem]{ulem}

%% Working with russian language
% \defaultfontfeatures{Ligatures={TeX},Renderer=Basic}
% \setmainfont[Ligatures={TeX,Historic}]{Times New Roman}
% \setmainfont{Times New Roman}
% \setsansfont{Comic Sans MS}
% \setmonofont{Courier New}

%% Russian Beamer
% \newtheorem{rtheorem}{Теорема}
% \newtheorem{rproof}{Доказательство}
% \newtheorem{rexpamle}{Пример}

%% Exta work with math
\usepackage{amsmath}
\usepackage{mathtools}
\usepackage{amssymb}
\usepackage{amsfonts}
\usepackage{amsthm}
\usepackage{icomma}
\usepackage{euscript}
\usepackage{mathrsfs}
\usepackage{gensymb}


%%
\newcommand*{\hm}[1]{#1\nobreak\discretionary{}{\hbox{$\mathsurround=Opt #1$}}{}}
\newcommand*{\romannum}[1]{\uppercase\expandafter{\romannumeral #1\relax}}

%% Images
\usepackage{graphicx}
\graphicspath{%
	{images}
}
\setlength\fboxsep{3pt}
\setlength\fboxrule{1pt}
\usepackage{wrapfig}
\usepackage[lflt]{floatflt}
\usepackage{subcaption}
\usepackage{caption}
\renewcommand\thesubfigure{\asbuk{subfigure}}
\usepackage{wrapfig} % Обтекание рисунков и таблиц текстом

%% Tables
\usepackage{array,tabularx,tabulary,booktabs}
\usepackage{longtable}
\usepackage{multirow}
\usepackage{float}
\restylefloat{table}

%% Programming
\usepackage{etoolbox}

\usepackage{lastpage}
\usepackage{soul}
\usepackage{csquotes}
\usepackage{multicol}
\usepackage{tikz}
\usepackage{pgfplots}
\usepackage{pgfplotstable}

%% Список источников
%\usepackage[style=gost-authoryear-min,maxcitenames=2,backend=biber,sorting=none]{biblatex}
\usepackage[%
backend=biber,% движок
bibencoding=utf8,% кодировка bib файла
sorting=none,% настройка сортировки списка литературы
style=gost-numeric,% стиль цитирования и библиографии (по ГОСТ)
language=autobib,% получение языка из babel/polyglossia, default: autobib % если ставить autocite или auto, то цитаты в тексте с указанием страницы, получат указание страницы на языке оригинала
autolang=other,% многоязычная библиография
clearlang=true,% внутренний сброс поля language, если он совпадает с языком из babel/polyglossia
defernumbers=true,% нумерация проставляется после двух компиляций, зато позволяет выцеплять библиографию по ключевым словам и нумеровать не из большего списка
sortcites=true,% сортировать номера затекстовых ссылок при цитировании (если в квадратных скобках несколько ссылок, то отображаться будут отсортированно, а не абы как)
doi=false,% Показывать или нет ссылки на DOI
isbn=false,% Показывать или нет ISBN, ISSN, ISRN
]{biblatex}

\addbibresource{include/coursework/references.bib}
\DefineBibliographyStrings{russian}{%
	bibliography = {Список источников},
}

\usepackage{pdfpages}



\setcounter{tocdepth}{1}
\usepackage[shortcuts]{extdash}
\usepackage{verbatim}
\usepackage{moreenum}

%% Page
\usepackage{extsizes}
\usepackage{geometry}
\geometry{top=20mm}
\geometry{bottom=30mm}
\geometry{left=15mm}
\geometry{right=15mm}

%% Page headers
\usepackage{fancyhdr}
\pagestyle{fancy}
% \renewcommand{\headrulewidth}{0pt}
\lhead{Лазба Филипп, РТ-42}
\rhead{}

\usepackage{xcolor}
\definecolor{light-gray}{gray}{0.8}
\newcommand{\code}[1]{\colorbox{light-gray}{\texttt{#1}}}

%% Line spacing
\usepackage{setspace}
\onehalfspacing

\usepackage{hyperref}
\hypersetup{%
    colorlinks=true,
    linkcolor=blue,
    filecolor=magenta,
    urlcolor=cyan,
    pdftitle={Курсовая работа  по курсу <<Спецразделы антенно-Фидерных устройств>>},
    bookmarks=true,
    pdfpagemode=FullScreen,
}

% Вложенные файлы
\usepackage{attachfile}



%%
\renewcommand{\epsilon}{\ensuremath{\varepsilon}}
\renewcommand{\phi}{\ensuremath{\varphi}}
\renewcommand{\kappa}{\ensuremath{\varkappa}}
\renewcommand{\le}{\ensuremath{\leqslant}}
\renewcommand{\leq}{\ensuremath{\leqslant}}
\renewcommand{\ge}{\ensuremath{\geqslant}}
\renewcommand{\geq}{\ensuremath{\geqslant}}
\renewcommand{\emptyset}{\varnothing}

\newcommand{\bhline}[1]{\noalign{\hrule height #1pt}}

\usepackage{titlesec}
\titleformat{\chapter}{\LARGE\normalfont\bfseries\filcenter}{}{1em}{}
\titlespacing{\chapter}{0mm}{0mm}{5mm}

\titleformat{\section}{\Large\normalfont\bfseries\filcenter}{}{1em}{}
\titleformat{\subsection}{\large\normalfont\bfseries\filcenter}{}{1em}{}
\titleformat{\subsubsection}{\normalfont\bfseries\filcenter}{}{1em}{}
